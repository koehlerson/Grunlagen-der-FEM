%=========================================================================
\sect{Introduction}
\label{Section1}
%=========================================================================
Due to the rapid developments in the field of computer technology
and the methods of calculation, respectively, the application
of numerical simulations becomes more and more significant
in all areas of engineering and research.
Generally, typical mechanical problems are examined as follows:

\begin{enumerate}
\item An appropriate mathematical model of the problem 
is derived. 
Usually, this results in a continuous problem, 
which is for example formulated in the framework of 
continuum mechanics. 
The problem is then described by a system of (partial) 
differential equations. 

\item The continuous system is approximated 
by an appropriate discrete problem. 
There the field variables, which have to be computed, 
are approximated by a finite number of values. 
This process is referred to as disretization; 
in this context we distinguish between disretization 
of the physical space and disretization of the system of 
differential equations. 

\item As a result of the discretization process we obtain 
algebraic systems of equations, which are solved by 
direct or iterative algorithms. 

\item After the solution of the discrete problem often 
several millions of computed quantities are available, 
which have to be visualized for interpretation. 

\end{enumerate}

Well-known discretization methods are: 
\begin{itemize}
\item Finite-Difference-Method
\item Finite-Element-Method
\item Finite-Volume-Method
\item Boundary-Element-Method. 
\end{itemize}

In the lecture ``Finite Element Method: Foundations'' we 
are dealing with the Finite-Difference-Method and the 
Finite-Element-Method for linear problems. 
In the field of solid mechanics the Finite-Element-Method 
is of particular significance due to its flexibility 
enabling the handling of complex problems for the 
numerical solution of elliptic and parabolic problems. 
Compared to the Finite-Difference-Method and the 
Finite-Volume-Method the Finite-Element-Method is 
specific to the variational formulation of differential 
equations. 

%%% Local Variables: 
%%% mode: latex
%%% TeX-master: t
%%% End: 
