\ssect{Plate Elements}

Plates carry off the main part of the loading via bending 
(cf. beams) and have a great technical importance. 
They are geometrically described by the thickness-measurement 
$h$ which is much smaller than the length-measurement 
$l_x$, $l_y$, 
i.e.
\eb
h \ll l_x, l_y \, .
\ee
We differentiate between shear-rigid and shear-weak 
formulations. 
If we start from the assumption that the cross-section 
planes remain even, we obtain the shear-weak formulation 
by {\sc Reissner \& Mindlin}, respectively. 
In this case, independent approximations for the deflection 
$w(x)$ and the curvature of the cross-section $\beta_x(x)$, 
$\beta_y(x)$ are considered. 
If we take into account the additional assumption of 
remaining normals (``normal remains normal''), we arrive at 
the shear-rigid theory of the {\sc Kirchhoff}-plate. 
In this case, $C^1$-consistent approximations for $w(x)$ 
are required, because the slope of the cross-section is 
computed by the derivatives $\beta_y=-w_{,y}$ and 
$\beta_x=-w_{,x}$. 
Hence, $\beta_x$ and $\beta_y$ are independent variables. 

%-------------------------------------------------------------------------
\sssect{Kinematics}
%-------------------------------------------------------------------------
\begin{Figure}[htb]
\begin{center}
\input{\dir/fig57.pstex_t}
\setlength{\baselineskip}{11pt}
\caption{Kinematics.}
\end{center}
\end{Figure}%

The deformation in the $x$-$z$-plane is given by the twist of the cross-section %Die Deformation in der $x$-$z$- Ebene ist durch die Verdrehung des Querschnitts
\eb
w_{,x} = - \beta_x + \gamma_{xz} \, .
\ee
Hence, the average shear strain results in
%bestimmt. Hieraus ergibt sich die mittlere Schubverzerrung zu
\eb
\gamma_{xz} = w_{,x} + \beta_x \, .
\ee
The $u$-displacement in $x$-direction is given via 
the curvature of the cross-section $\beta_x$
\eb
u = \beta_x z \, .
\ee
Analogously, the deformation in the $y$-$z$-plane and therefore the twist of the cross-section results in
\eb
w_{,y} = - \beta_y + \gamma_{yz} \, .
\ee
Hence, we obtain the average shear strain
\eb
\gamma_{yz} = w_{,y} + \beta_y \, .
\ee
The $v$-displacement in $y$-direction reads
%Die $v$- Verschiebung in $y$- Richtung ist
\eb
v = \beta_y z \, .
\ee

\begin{Figure}[htb]
\begin{center}
\input{\dir/fig59.pstex_t}
\setlength{\baselineskip}{11pt}
\caption{Stress distribution.} 
\end{center}
\end{Figure}%
\begin{Figure}[htb]
\begin{center}
\input{\dir/fig58.pstex_t}
\setlength{\baselineskip}{11pt}
\caption{Internal force variables.}
\end{center}
\end{Figure}%

%-------------------------------------------------------------------------
\sssect{Kirchhoff Theory}
%-------------------------------------------------------------------------
With respect to the shear-rigid {\sc Kirchhoff}-theory the 
assumption of remaining normals is evaluated and 
%Im Rahmen der schubstarren {\sc Kirchhoff}- Theorie wird das Senkrechtbleiben der Querschnitte gefordert und es gilt
\eb
\gamma_{xz} = \gamma_{yz} = 0 \, 
\ee
is aplied. Hence, for the curvatures of the cross-sections 
it follows that 
%Hieraus folgt f\"ur die Querschnittsneigungen
\eb
\beta_x = - w_{,x} \qquad \mbox{und} \qquad \beta_y = - w_{,y}\, ,
\ee
as well as for the displacements
%sowie f\"ur die Verschiebungen
\eb
u = - z w_{,x} \qquad \mbox{and} \qquad v = - z w_{,y} \, .
\ee
For the strains we obtain
%F"ur die Verzerrungen erhalten wir
\eb
\Bvarepsilon = \left[ \begin{array}{c} \varepsilon_x \\ \varepsilon_y \\
\gamma_{xy} \end{array} \right] = \left[ \begin{array}{c} u_{,x} \\
v_{,y} \\ u_{,y} + v_{,x} \end{array} \right] =
\left[ \begin{array}{c} - z w_{,xx} \\ -z w_{,yy} \\ -z (w_{,xy} +
w_{,yx}) \end{array} \right] =
z \left[ \begin{array}{c} - w_{,xx} \\ - w_{,yy} \\ - 2 w_{,xy}
\end{array} \right] = z \Bkappa \, ,
\label{eq:nKinStr}
\ee
whereby $\Bkappa$ characterizes the curvatures. With the material law
%wobei $\Bkappa$ die Kr"ummungen charakterisiert. Mit dem Stoffgesetz
\eb
\left[ \begin{array}{c} \sigma_x \\ \sigma_y \\ \tau_{xy} \end{array}
\right] = \dfrac{E}{1 - \nu^2} \left[ \begin{array}{ccc} 1 & \nu & 0 \\
\nu & 1 & 0 \\ 0 & 0 & \dfrac{1 - \nu}{2} \end{array} \right]
\left[ \begin{array}{c} \varepsilon_x  \\ \varepsilon_y \\ \gamma_{xy}
\end{array} \right] \quad \mbox{and} \; \Bsigma = {\bf \IC} \,
\Bvarepsilon, \quad \mbox{respectively} \, ,
\label{eq:nMatLaw}
\ee
we are able to calculate the stress resultants by the 
integration over the thickness
%k"onnen wir nun die Schnittgr"o"sen "uber Dickenintegration berechnen
\eb
m_x = \int_{(z)} \sigma_x z \, dz \, , \quad m_y = \int_{(z)} \sigma_y z
\, dz \, , \quad m_{xy} = \int_{(z)} \tau_{xy} z \, dz \, .
\label{eq:nSchnittGr}
\ee
With (\ref{eq:nMatLaw}) and (\ref{eq:nKinStr}) we obtain 
%Mit (\ref{eq:nMatLaw}) und (\ref{eq:nKinStr}) erhalten wir aus (\ref{eq:nSchnittGr})
\eb
\bM := \left[ \begin{array}{c} m_x \\ m_y \\ m_{xy} \end{array} \right] =
\int_{(z)} \Bsigma z \, dz = \int_{(z)} {\bf \IC} z^2 \Bkappa \, dz =
{\bf \IC} \int_{-h/2}^{h/2} z^2 \, dz \, \Bkappa
\label{eq:nSchnittGr2}
\ee
and therefore 
%und es folgt
\eb
\bM = \dfrac{E h^3}{12 (1 - \nu^2)} \left[ \begin{array}{ccc}
1 & \nu & 0 \\ \nu & 1 & 0 \\ 0 & 0 & \dfrac{1 - \nu}{2} \end{array}
\right] \Bkappa = : {\bf \IC}_B \Bkappa \, .
\ee
The principle of virtual work reads 
%Prinzip der virtuellen Arbeit
\eb
\delta\Pi = \delta\Pi_i + \delta\Pi_a
\ee
with the internal and external parts 
%mit
\eb
\delta\Pi_i = \int_{(A)} \delta\Bkappa^T \bM \, dA \qquad \mbox{and}
\qquad \delta\Pi_a = \int_{(A)} \delta w q_z \, dA \, .
\ee
In the linear elastic case, the potential of the inner forces reads
%Im linear elastischen Fall ist das Potential der inneren Kr"afte
\eb
\Pi_i = \dfrac{1}{2} \int_{(A)} \Bkappa^T {\bf \IC}_B \Bkappa \, dA \, .
\ee
The ansatz functions with respect to the {\sc Kirchhoff}- theory 
have to be continuous in the displacement and in the 
derivatives of the displacements, i.e. we require 
$C^1$-continuous functions. 
%Die Ansatzfunktionen im Rahmen der {\sc Kirchhoff}- Theorie m"ussen stetig in den Verschiebungen und den Ableitungen der Verschiebungen sein, d.h. wir ben"otigen $C^1$- stetige Funktionen.






%-------------------------------------------------------------------------
\sssect{16 Parameter Element by Kirchhoff- Theory}
%-------------------------------------------------------------------------
According to the {\sc Kirchhoff} theory with respect to a 
rectangular plate element, the approximations of the 
Bernoulli-beam with 4 Hermite polynomials 
%Bei dem Rechteckplattenelement nach der {\sc Kirchhoff}- Theorie werden die Ans"atze des Bernoulli- Balkens mit 4 Hermiten Polynomen auf die Platte "ubertragen. Beim Stab war
\eb
w(\xi) = N_1(\xi) w_1 + \bar{N}_1(\xi) \beta_1 + N_2(\xi) w_2 +
\bar{N}_2(\xi) \beta_2 \, 
\label{approxbeam}
\ee
are transferred to the case of a plate, thus, we require 
16 ansatz functions from a formal 2-dimensional expansion 
of (\ref{approxbeam}). 
The ansatz for $w_\nodeid{1}$ is given by 
%F"ur die Platte ben"otigen wir aus einer formalen 2- dimensionalen Erweiterung von () nun 16 Ansatzfunktionen. Der Ansatz f"ur $w_1$ ist
\eb
N_{11} = N_{\nodeid{1}}(\xi) N_{\nodeid{1}}(\eta) \, ,
\ee
and the approximations for $\beta_{y\nodeid{1}}$ and $\beta_{x\nodeid{1}}$ 
are
%die Ans"atze f"ur $\beta_{1,y}$ und $\beta_{1,x}$ sind
\eb
N_{1\bar{1}} = N_{\nodeid{1}}(\xi) \bar{N}_{\nodeid{1}}(\eta) \; ; 
\quad 
N_{\bar{1}1} = \bar{N}_{\nodeid{1}}(\xi) N_{\nodeid{1}}(\eta) \, . 
\ee
With the node displacement vector
%Mit dem Knotenverschiebungsvektor
\eb
\bd_I^T = [w_I, \beta_{xI}, \beta_{yI}, w_{xyI}] \quad \mbox{for} \;
I = 1, ..., 4
\ee
we obtain the element displacement vector
%erhalten wir den Elementverschiebungsvektor
\eb
\bd^{eT} = [\bd_{\nodeid{1}}^T, \bd_{\nodeid{2}}^T, \bd_{\nodeid{3}}^T, \bd_{\nodeid{4}}^T] \, .
\ee
From this we get the approximation for the deflection 
%Es folgt
\eb
w^h = \sum_{I=1}^{4} \bN_I \cdot \bd_I \, , 
\ee
wherein
%mit
\eb
\left. \begin{array}{ll}
\bN_{\nodeid{1}} & = [N_{11}, N_{\bar{1}1}, N_{1\bar{1}}, N_{\bar{11}}] \\
\bN_{\nodeid{2}} & = [N_{21}, N_{\bar{2}1}, N_{2\bar{1}}, N_{\bar{21}}] \\
\bN_{\nodeid{3}} & = [N_{22}, N_{\bar{2}2}, N_{2\bar{2}}, N_{\bar{22}}] \\
\bN_{\nodeid{4}} & = [N_{12}, N_{\bar{1}2}, N_{1\bar{2}}, N_{\bar{12}}]
\end{array} \right\}.
\ee
This ansatz is cubically complete with ansatz functions 
$N_I$, $\bar N_I$ following the scheme 
%Dieser Ansatz ist mit
\eb
\left. \begin{array}{ccccccc}
 & & & 1  & & & \\
 & & \xi & & \eta & & \\
 & \xi^2 & & \xi\eta & & \eta^2 & \\
\xi^3 & & \xi^2\eta & & \xi\eta^2 & & \eta^3 \\
 & \xi^3\eta & & \xi^2\eta^2 & & \xi\eta^3 & \\
 & & \xi^3\eta^2 & & \xi^2\eta^3 & & \\
 & & & \xi^3\eta^3 & & &
\end{array} \right\} \, . 
\ee
Herewith, we are able to represent rigid-body displacements 
and twists, constant curvatures and constant torsions. 
Furthermore, homogeneous differential equations 
(no loading) are exactly reflected if linear stress 
resultants and constant shear forces occur. 
The approximation for the strains is then given by 
%kubisch vollst"andig. Somit lassen sich Starrk"orperverschiebungen 
% und Verdrehungen $(1,x,y)$ und konstante Kr"ummungen $(x^2,y^2)$ 
% und konstante Verwindungen $(x,y)$ darstellen. 
% Zudem werden Zust"ande, die durch eine homogene DGL beschrieben 
% werden (keine Belastung) mit linear ver"anderlichen Momenten 
% und konstanten Querkraftverl"aufen exakt beschrieben. 
% Die Verzerrungen berechnen sich aus
\eb
\Bkappa^h = \left[ \begin{array}{c} - w_{,xx} \\ - w_{,yy} \\ - 2 w_{,xy}
\end{array} \right] = \sum_{I=1}^{4} \bB_I \bd_I
\ee
with the B-matrix
%mit
\eb
\bB_I = \left[ \begin{array}{c} - N_{I,xx} \\ - N_{I,yy} \\ - 2 N_{I,xy}
\end{array} \right]
\ee
and the particular terms 
%und
\eb
N_{I,xx} = N_{I,\xi\xi} \dfrac{4}{l_x^2} \; ; \quad
N_{I,yy} = N_{I,\eta\eta} \dfrac{4}{l_y^2} \; ; \quad
N_{I,xy} = N_{I,\xi\eta} \dfrac{4}{l_x l_y} \, .
\ee
The stiffness matrix is then computed by 
%Die Steifigkeitsmatrix ergibt sich aus
\eb
k_{IJ}^e = \int_{(A)} \bB_I^{eT} {\bf \IC}_B \bB_J^e \, dA \, .
\ee
The integration can be computed by numerical quadratures 
or analytically. 
%Die Integration kann sowohl analytisch als auch numerisch erfolgen.

%-------------------------------------------------------------------------
\sssect{Reissner-Mindlin Theory}
%-------------------------------------------------------------------------
With respect to the shear-weak formulation by {\sc Reissner, Mindlin} 
the displacements are computed by 
%Im Rahmen der schubweichen Formulierung nach 
% {\sc Rei"sner} bzw. {\sc Mindlin} ergibt sich f\"ur die 
% Verschiebungen
\eb
u = z \beta_x \qquad \mbox{and} \qquad v = z \beta_y \, . 
\ee
Then the strains are given by 
%Verzerrungen
\eb
\Bvarepsilon = \left[ \begin{array}{c} z \beta_{x,x} \\ z \beta_{y,y} \\
z (\beta_{x,y} + \beta_{y,x}) \end{array} \right] = z \Bkappa \;,
\ee
and a material law has to be provided also for the shear stresses 
which are not in the plate plain, i.e. 
%Materialgesetz
\eb
\left[ \begin{array}{c} \sigma_x \\ \sigma_y \\ \tau_{xy} \end{array}
\right] = \dfrac{E}{1 - \nu^2} \left[ \begin{array}{ccc} 1 & \nu & 0 \\
\nu & 1 & 0 \\ 0 & 0 & \dfrac{1 - \nu}{2} \end{array} \right]
\left[ \begin{array}{c} \varepsilon_x \\ \varepsilon_y \\ \gamma_{xy}
\end{array} \right] \; ; \quad \left[ \begin{array}{c} \tau_{xz} \\
\tau_{yz} \end{array} \right] = \kappa_s \,\mu\, {\bf 1} \left[
\begin{array}{c} \gamma_{xz} \\ \gamma_{yz} \end{array} \right]
\ee
which can be expressed in the total notation by 
%bzw.
\eb
\Bsigma = {\bf \IC}\, \Bvarepsilon \qquad \mbox{and} \qquad
\Btau = \bG \Bgamma \, , 
\ee
respectively. 
Herein, $\kappa_s$ is the shear correction coefficient. 
In addition to the stress resultants $m_x$, $m_y$, $m_{xy}$ 
in (\ref{eq:nSchnittGr2}), we consider 
%$\kappa_s$ ist der Schubkorrekturfaktor. Zus"atzlich zu den Schnittgr"o"sen $m_x$, $m_y$, $m_{xy}$ nach (\ref{eq:nSchnittGr2}) erhalten wir
\eb
q_{xz} = \int_{(z)} \tau_{xz} \, dz 
\qquad \mbox{and} \qquad 
q_{yz} = \int_{(z)} \tau_{yz} \, dz 
\ee
and obtain the additional stress resultants 
%Schnittgr"o"sen
\eb
\bQ = \left[ \begin{array}{c} q_{xz} \\ q_{yz} \end{array} \right] =
\int_{-h/2}^{h/2} \bG \Bgamma \, dz = \kappa_s\, \mu\, h \left[
\begin{array}{cc} 1 & 0 \\ 0 & 1 \end{array}\right] \Bgamma =
{\bf \IC}_s \Bgamma \, . 
\ee
The principle of virtual work states that 
%Prinzip der virtuellen Arbeit
\eb
\delta\Pi = \delta\Pi_i + \delta\Pi_a \, , 
\ee
with the internal 
%mit
\eb
\delta\Pi_i = \int_{(A)} \delta\Bkappa^T \bM \, dA +
\int_{(A)} \delta\Bgamma^T \bQ \, dA
\ee
and external part
%und
\eb
\delta\Pi_a = \int_{(A)} \delta w q_z \, dA \, . 
\ee
The internal potential reads then 
%Potential
\eb
\Pi_i = \dfrac{1}{2} \int_{(A)} \Bkappa^T {\bf \IC}_B \Bkappa \, dA +
\dfrac{1}{2} \int_{(A)} \Bgamma^T {\bf \IC}_S \Bgamma \, dA
\ee

%-------------------------------------------------------------------------
\ssect{Isoparametric Four-Node-Element of the Shear-Elastic
Theory}
%-------------------------------------------------------------------------
The degrees of freedom at the nodes are $w$, $\beta_x$, $\beta_y$. 
Thus, we consider the approximation 
\eb
\bu^h = \left[ \begin{array}{c} w \\ \beta_x \\ \beta_y \end{array}
\right] = \sum_{I=1}^{n_{el}} N_I \bd_I \;,
\ee
with the ansatz functions
%Ansatzfunktionen
\eb
N_I = \dfrac{1}{4} (1 + \xi \xi_I) (1 + \eta \eta_I)\, . 
\ee
The approximations for the curvatures are given by 
%Kr"ummungen
\eb
\Bkappa^h = \left[ \begin{array}{c} \beta_{x,x}^h \\ \beta_{y,y}^h \\
\beta_{x,y}^h + \beta_{y,x}^h \end{array} \right] = \sum_{I=1}^{n_{el}}
\bB_b^I \bd_I 
\quad \mbox{with} \quad 
\bB_b^I = \left[ \begin{array}{cc}
N_{I,x} & 0 \\ 0 & N_{I,y} \\ N_{I,y} & N_{I,x} \end{array} \right] \, , 
\ee
whereas the shear-strains are approximated by 
%Schubverzerrungen
\eb
\gamma^h = \left[ \begin{array}{c} w_{,x}^h + \beta_x^h \\ 
w_{,y}^h + \beta_y^h\end{array} \right]  
= \sum_{I=1}^{n_{el}} \bB_s^I \bd_I 
\quad \mbox{with}\quad 
\bB_s^I = \left[ \begin{array}{ccc} N_{I,x} & N_I & 0 \\ N_{I,y} & 0 &
N_I \end{array} \right]\, . 
\ee
The Jacobi matrix is computed by 
%Jakobi Matrix
\eb
\left[ \begin{array}{c} N_{I,x} \\ N_{I,y} \end{array} \right] =
\bJ^{-1} \left[ \begin{array}{c} N_{I,\xi} \\ N_{I,\eta} \end{array}
\right] \, , 
\ee
hence, we obtain the element stiffness matrix decomposed into 
two parts by 
\eb
\delta\Pi_i^e = \delta\bd^{eT} \left[ \underbrace{\int_{(A)} \bB_b^{eT}
{\bf \IC}_b \bB_b^e \, dA}_{=: \bk_b^e} + \underbrace{\int_{(A)}
\bB_s^{eT} {\bf \IC}_s \bB_s^e \, dA}_{=: \bk_s^e} \right] \bd^e \, . 
\ee


