\ssect{Quadrilateral Element}

This section describes the formal derivation of the 
variational problem arising in linear elasticity and 
its Finite-Element discretization in the context of the 
isoparametric concept. 

%%%%%%%%%%%%%%%%%%%%%%%%%%%%%%%%%%%%%%%%%%%%%%%%%%%%%%%%%%%%
\sssect{Variational Formulation}
%%%%%%%%%%%%%%%%%%%%%%%%%%%%%%%%%%%%%%%%%%%%%%%%%%%%%%%%%%%%

In this section we deduce the weak form of the boundary 
value problem. In the framework of linear elastostatics our 
body of interest $\mathcal{B}\in\mathcal{R}^3$ is 
parametrized in $\bx\in\mathcal{B}$.
Starting with the balance of linear momentum 
we consider firstly the strong form of the boundary value 
problem
%
\eb
\div[\Bsigma] + \bar\bbf = \bzero\ \forall\ \bx\in\mathcal\B \, , 
\ee
%
wherein $\Bsigma$ and $\bar\bbf$ denote the Cauchy-stress 
matrix and the volume force vector, respectively. 
In the context of the Finite-Element Method we take into 
account cartesian coordinates only, therefore 
we skip the base vectors of all tensorial coefficients 
and use matrix notation. 
We concentrate on small strains, thus, the strain matrix 
is defined as the symmetric part of the displacement gradient 
%
\eb
\Bvarepsilon := \dfrac{1}{2} [ \nabla \bu + \nabla^T \bu] \, .
\ee
%
For the complete description of the boundary value 
problem we need boundary conditions
%
\eb
\bu = \bar{\bu} \ \mbox{on} \ \partial \mathcal{B}_u
\quad\mbox{and}\quad
\bt = \Bsigma \cdot \bn = \bar{\bt} \ \mbox{on} \ \partial \mathcal{B}_{\sigma}.
\ee

\begin{Figure}[h] \unitlength 1 cm
\begin{picture}(11,5.)
\put(0.1,-.40){\scalebox{1.25}{\input{\dir/fig5_04.pstex_t}}}
\end{picture}
\caption{Illustration of the boundary value conditions}
\label{fig5_04}
\end{Figure}

In order to obtain a principally solvable partial differential 
equation we apply the Galerkin method. 
Thus, we multiply the strong form with a test function 
${\delta\bu}$ and integrate over the domain ${\mathcal\B}$ 
and obtain 
%
\eb
G := - \int_{\mathcal{B}} (\div[{\Bsigma]}\cdot \delta \bu + \bar{\bbf} \cdot \delta \bu )dv= 0 .
\label{weakform1}
\ee
%
Reformulating ${\div\Bsigma \cdot \delta \bu}$ and applying 
the divergence theorem leads to
%
\eb
\left.
\begin{array}{rl}
\sigma_{ij,j} \delta u_i &= (\sigma_{ij} \delta u_i)_{,j} - \sigma_{ij}\delta u_{i,j}\\
\\
&=\div[\Bsigma]\cdot\delta\bu+\Bsigma \cdot \mbox{grad}[\delta\bu]\\
\\
\Leftrightarrow\div[\Bsigma]\cdot\delta\bu &=\div[\Bsigma\,\delta\bu]-\Bsigma \cdot \mbox{grad}[\delta\bu]
\end{array}
\right\}.
\ee
%
Inserting this expression into equation (\ref{weakform1}) 
leads to 
%
\eb
G = - \int_{\mathcal{B}} \div[\Bsigma\,\delta\bu] dv
       + \int_{\mathcal{B}} \mbox{grad}[\delta \bu] \, \cdot \, \Bsigma dv
      - \int_{\mathcal{B}} \delta \bu \cdot \bar{\bbf} dv = 0 \, . 
\ee
%
Applying the Gauss integral and considering the Cauchy theorem ${\Bsigma\,\bn=\bt}$ we obtain
%
\eb
- \int_{\mathcal{B}} \div [\Bsigma\,\delta \bu] dv = 
- \int_{\partial \mathcal{B}} \delta \bu \cdot (\Bsigma \, \bn) da =
- \int_{\partial \mathcal{B}} \delta \bu \cdot \bt \; da \, . 
\ee
%
Setting this relation into the previous equation yields
%
\eb
G =\int_{\mathcal{B}} \Bsigma\cdot\mbox{grad}[\delta\bu]
    - \int_{\mathcal{B}}\delta \bu \cdot \bar{\bbf} dv -
       \int_{\partial \mathcal{B}_{t}} \delta \bu \cdot \bar{\bt}
       da = 0 \, . 
\ee
%
Now we compute the virtual strains which are given by 
%
\eb
\delta\Bvarepsilon= \dfrac{1}{2}(\mbox{grad}[\delta\bu]+\mbox{grad}^T[\delta\bu])= \mbox{grad}^{sym}[\delta \bu] \, .
\ee
%
Evaluation of the balance of angular momentum and 
consideration of the local form leads to the symmetry of the 
stress matrix
%
\eb
\Bsigma= \Bsigma^T
\ee
%
and therefore the equation holds
%
\eb
\Bsigma\cdot\mbox{grad}[\delta\bu] 
= \Bsigma\cdot\mbox{grad}^{sym}[\delta\bu] 
= \Bsigma\cdot\delta\Bvarepsilon \, .
\ee
%
Finally we obtain the weak form of the boundary value problem
%
\eb
G = \underbrace{\int_{\mathcal{B}} \delta \Bvarepsilon : 
       \Bsigma dv}_{G_{int}}
    - \underbrace{ \int_{\mathcal{B}} \delta \bu \cdot \bar{\bbf} dv -
       \int_{\partial \mathcal{B}_{t}} \delta \bu \cdot \bar{\bt} \;
       da}_{G_{ext}} = 0 \, , 
\ee
%
wherein $G_{int}$ and $G_{ext}$ represent the internal and 
external part, respectively. 











\newpage
%%%%%%%%%%%%%%%%%%%%%%%%%%%%%%%%%%%%%%%%%%%%%%%%%%%%%%%%%%%%
\sssect{Finite-Element-Discretization}
%%%%%%%%%%%%%%%%%%%%%%%%%%%%%%%%%%%%%%%%%%%%%%%%%%%%%%%%%%%%

In the context of the Finite-Element-Method is based on 
the approximation of the primary variables 
%
\eb
\bu^h (\Bxi) = \sum_{I=1}^{n_{ele}} N_I (\Bxi) \bd_I .
\label{approxu}
\ee
%
Herein, $\Bxi = [\xi,\eta]^T$ denotes the vector of 
parametric coordinates. 
With view to the isoparametric concept the geometry 
is approximated in the same way 
%
\eb
\bx^h (\Bxi) = \sum_{I=1}^{n_{ele}} N_I (\Bxi) \bx_I \, .
\label{approxx}
\ee
%
Please note, that superscript $h$ denotes approximated 
quantities. 
Exemplarily we consider the discretization of a typical 
2D- four node isoparametric element. 
The transformation of the parametric domain into the 
physical domain can be interpreted as a transformation from 
the reference configuration to the actual configuration, 
see Fig. \ref{fig507}.

\begin{Figure}[htb]
\begin{center}
\input{\dir/fig5_7.pstex_t}
\setlength{\baselineskip}{11pt}
\caption{Illustration of the transformatiom from the 
parametric domain into the physical domain. }
\label{fig507}
\end{center}
\end{Figure}

The bilinear ansatz functions $\bN_I$ in (\ref{approxu}) 
and (\ref{approxx}) interpolate between the quantities at 
the nodes of the element and are given for quadrilateral 
elements by
%
\eb
N_I(\xi,\eta) = \dfrac{1}{4} (1 + \xi \xi_I)(1 + \eta \eta_I) \, . 
\ee
%
If we insert the parametric coordinates of the nodes 
we obtain the explicit ansatz functions 
%
\eb
\left.
\begin{array}{rl}
   N_1 = \dfrac{1}{4} (1-\xi)(1-\eta)\\
\\ N_2 = \dfrac{1}{4} (1+\xi)(1-\eta)\\
\\ N_3 = \dfrac{1}{4} (1+\xi)(1+\eta)\\
\\ N_4 = \dfrac{1}{4} (1-\xi)(1+\eta)
\end{array}
\right\}.
\ee

In Fig. \ref{fig507} you see a graphical illustration 
of the bilinear ansatz functions 

\begin{Figure}[htb]
\begin{center}
\input{\dir/ansatz_nen4.pstex_t}
\setlength{\baselineskip}{11pt}
\caption{Bilinear ansatz functions for a four node element }
\end{center}
\end{Figure}

Due to the fact that the stress and strain matrices are 
symmetric, it is not necessary to consider all 9 coefficients. 
For this purpose we now switch to the reduced matrix notation, 
where the stress and strain matrix are denoted as vectors. 
Then we obtain the relation 
%
\eb
\Bsigma = 
\left[ \begin{array}{c}
\sigma_{xx}\\
\sigma_{yy}\\
\sigma_{xy}
\end{array} \right] = \dfrac{E}{(1+\nu)(1-2\nu)}
\left[ \begin{array}{ccc}
    1 & \nu & 0 \\
    \nu & 1 & 0 \\
    0 & 0 & \dfrac{1-2\nu}{2}
\end{array} \right]
\left[ \begin{array}{c}
\varepsilon_{xx}\\
\varepsilon_{yy}\\
2\varepsilon_{xy}
\end{array} \right] 
= \IC\,\Bvarepsilon , 
\label{elasticity}
\ee
%
wherein the elasticity matrix is denoted by $\IC$. 
Since the strains are expressed by the symmetric part of 
the displacement gradient 
$\Bvarepsilon := \dfrac{1}{2} (\mbox{grad}[\bu] + \mbox{grad}^T[\bu])$ 
we obtain 
%
\eb
\Bvarepsilon =
\left[ \begin{array}{c}
\varepsilon_{xx}\\
\varepsilon_{yy}\\
\gamma_{xy}
\end{array} \right] =
  \left[ \begin{array}{c}
  u_{x,x}\\
  u_{y,y}\\
  u_{x,y}+u_{y,x}
\end{array} \right] , 
\label{gradu}
\ee
%
and note that we are able to express the stresses via 
derivation of the displacements with respect to $\bx$. 
Herein, the displacements $\bu$ have to be approximated by 
%
\eb
\bu^h=\sum_{I=1}^{4}N_I\bd_I=\bN^e\bd^e 
\ee
%
with $\bN$ being the ansatzfunction matrix defined by
%
\eb
\left[ \begin{array}{cc}
       u_x^h\\ u_y^h
       \end{array}\right]=
\left[ \begin{array}{cc|cc|cc|cc}
         N_{1} &0& N_2&0 &N_{3}&0&N_{4}&0 \\
         0& N_{1}& 0&N_{2}&0&N_{3}&0&N_{4} \\

\end{array} \right]
\left[ \begin{array}{c}
d_{1x}\\
d_{1y}\\ \hline
d_{2x}\\
d_{2y}\\ \hline
d_{3x}\\
d_{3y}\\ \hline
d_{4x}\\
d_{4y}\\
\end{array} \right] .
\ee
%
In the same way we are able to compute the approximation of the virtual displacements
%
\eb
\delta\bu^h = \sum_{I=1}^{4}N_I(\xi)\delta\bd_I 
          = \bN^e\delta\bd^e \, .
\ee
%
For the approximation of all solution fields occurring in 
the weak form of equilibrium we need the approximation 
of strains, thus, we insert the displacement 
approximations into (\ref{gradu}) and obtain 
%
\eb
\Bvarepsilon^h = \sum_{I=1}^{4}N_{I,x} (\Bxi)\bd_I 
             = \sum_{I=1}^4 \bB_I (\Bxi) \bd_I \, . 
\ee
%
Herein, we introduced the $\bB_I$-matrix at the nodes given by 
%
\eb
\bB_I = \left[ \begin{array}{cc}
        N_{I,x} & 0\\
        0 & N_{I,y} \\
        N_{I,y} & N_{I,x} 
\end{array} \right] \, . 
\label{Bmatrix}
\ee
%
This notation can be abbreviated by the introduction of 
a $\bB^e$-matrix per element following 
%
\eb
\Bvarepsilon^h=\bB^e\bd^e = 
\left[ \begin{array}{cc|cc|cc|cc}
         N_{1,x} &0&N_{2,x}& 0 &N_{3,x}&0&N_{4,x}&0 \\
         0& N_{1,y}& 0&N_{2,y}&0&N_{3,y}&0&N_{4,y} \\
         N_{1,y} & N_{1,x} &N_{2,y} & N_{2,x}&N_{3,y} & N_{3,x}&N_{4,y}&N_{4,x}\\
\end{array} \right]
\left[ \begin{array}{c}
d_{1x}\\
d_{1y}\\ \hline
d_{2x}\\
d_{2y}\\ \hline
d_{3x}\\
d_{3y}\\ \hline
d_{4x}\\
d_{4y}\\
\end{array} \right] .
\ee
%
Analogously we obtain for the virtual strains the 
approximation
%
\eb
\delta\Bvarepsilon^h = \sum_{I=1}^4 \bB_I \delta\bd_I\\
                   = \bB^e\delta\bd^e \, .
\ee
%
Due to the fact that the ansatz functions depend on 
the parametric coordinates $\Bxi$, we have to apply 
the chain rule for the computation of the derivatives 
of $N$ with respect to $\bx$ in (\ref{Bmatrix}) and 
obtain 
%
\eb
\dfrac{\partial N_I}{\partial \bx} = 
\dfrac{\partial N_I}{\partial \Bxi}\,
\dfrac{\partial \Bxi}{\partial \bx} = 
\dfrac{\partial N_I}{\partial \Bxi}\,\bJ^{-1} \, . 
\label{derivativeofN}
\ee
%
Herein we introduced the inverse of the Jacobi matrix $\bJ$. 
The Jacobi matrix $\bJ =\dfrac{\partial\bx}{\partial\Bxi}$ 
transforms quantities in the parametric to the physical 
coordinate system. 
The partial derivatives occurring in (\ref{derivativeofN}) 
can be computed by
%
\eb
\renewcommand{\arraystretch}{2.0}
\left[ \begin{array}{c}
        N_{I,\xi}\\
        N_{I,\eta}
\end{array} \right] =
\left[ \begin{array}{c}
\dfrac{1}{4} \xi_I (1+\eta \eta_I)\\
\dfrac{1}{4} \eta_I (1+\xi \xi_I)
\end{array} \right] \, .
\ee
%
Inserting (\ref{elasticity}) and all approximated fields 
into the weak form we obtain the approximated weak form 
for one finite element
%
\eb
\left.
\begin{array}{rl}
G_e \approx G_e^h &= \displaystyle\int_\B\bB^e\delta\bd^e\IC\bB^e\bd^e \,dv 
- \int_\B\bN^e\delta\bd^e\cdot\bff \,dv 
+ \int_{\partial\B}\bN^e\delta\bd^e\cdot\bt \,da\\\\
&= \delta\bd^{eT}\left[ \begin{array}{c} \underbrace{\int_{\B}\bB^{eT}\IC\bB^edv}_{\bk^e}\bd^e-\underbrace{\int_\B\bN^{eT}\bbf dv+\int_{\partial\B}\bN^{eT}\bt da}_{\br^e}
\end{array} \right]\\
\\
&=(\delta\bd^{eT})\bk^e\bd^e-\br^e=0
\end{array}
\right\} .
\ee
%
Herein $\bk^e$ denotes the element stiffness matrix and 
$\br^e$ is the abbreviation for the element force vector. 





















