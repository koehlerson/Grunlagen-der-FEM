\clearpage
\setcounter{page}{1}

\section{Ansatzfunktionen}


\subsection{Verständnisfragen}

\enab
\item Wie lassen sich grundsätzlich FE-Elementansatzfunktionen bestimmen?
\item Welche Eigenschaft der Ansatzfunktionen wird dabei genutzt?
\item Was ist die Idee des isoparametrischen Konzeptes? 
\item Welche Vorteile hat es die FE-Approximation über das Referenzelement im Parameterraum durchuführen?
\enae

\subsection{Isoparametrisches Viereckselement}


{\center\begin{minipage}[t]{0.36\textwidth}
\input{fig/ue8_q1.pdf_tex}
\end{minipage}
\begin{minipage}[t]{0.36\textwidth}
\input{fig/ue8_q2.pdf_tex}
\end{minipage}
\captionof{figure}{lineares und quadratisches Referenzviereckselement im Parameterraum}
\label{fig:qiso}
}

\enab
\item Bestimmen Sie die Ansatzfunktionen $N_1$ und $N_2$ für das lineare Viereckselement aus Abbildung \ref{fig:qiso}
\item Bestimmen Sie die Ansatzfunktionen $N_2$ und $N_6$ für das quadratische Viereckselement aus Abbildung \ref{fig:qiso}
\enae


\clearpage
\subsection{Referenzdreieckselement}


{\center\begin{minipage}[t]{0.36\textwidth}
\input{fig/ue8_p1.pdf_tex}
\end{minipage}
\begin{minipage}[t]{0.36\textwidth}
\input{fig/ue8_p2.pdf_tex}
\end{minipage}
\captionof{figure}{lineares und quadratisches Referenzdreieckselement im Parameterraum}
\label{fig:piso}
}


\enab
\item Bestimmen Sie die Ansatzfunktionen für das lineare Dreieckselement aus Abbildung \ref{fig:piso} mithilfe des allgemeinen Ansatzes
\item Wie lauten die Ansatzfunktionen $N_1$ und $N_6$ für das quadratische Dreieckselement aus Abbildung \ref{fig:piso}. ?
\item Wie lauten die Ansatzfunktionen für ein dreidimensionales lineares Tetraederelement?
\enae