\clearpage
\setcounter{page}{1}

\section{Gauss-Quadratur}

Die Formel für die numerische Integration mittels Gauss-Quadratur lautet
\eb
\int_{-1}^1 f(\xi)\ \mrm{d} \xi \approx
\sum_{i=1}^n w_i\ f(\xi_i)\ .
\ee
Dabei sind $w_i$ die Wichungsfaktoren und $\xi_i$ die Gausspunkte.


\subsection{Quadratisches Viereckselement}

Für das quadratische isoparametrische Viereckselement wird die Gaussintegrationsordnung $n=3$ gewählt.

\enab
\item Wie viele Gausspunkte hat das Element dann? 
\item Ist die Gaussintegration der Ordnung $n=3$ für eine quadratische Funktion $f(\xi)$ (Polynomgrad $p=2$) exakt? 
      Welche Bedingung muss erfüllt sein?
          
      \textit{Hinweis}: Da bei Viereckselementen die Übertragung von einer auf zwei Koordinatenrichtungen direkt möglich ist reicht es in dieser Aufgabe eine Funktion $f(\xi)$ mit nur einer Koordinate zu betrachten.
\enae




\enabres
\item Gibt es Gründe eine höhere Integrationsordnung als nötig zu verwenden? Erläutern Sie ihre Antwort anhand eines Beispiels.
\enae

Findet die Gaussintegration über das Intervall $[-1,1]$ statt so lassen sich die Gausspunke $\xi_i\ (i=1,\hdots,n)$ als Nullstellen des n-ten Legendrepolynoms bestimmen:

\eb
P_n(\xi)=\frac{1}{2^n\,n!}\ \frac{\mrm{d}^n}{\mrm{d}\xi^n}(\xi^2-1)^n
\ee

\enabres
\item Wie lautet das Legendrepolynom 3. Ordnung?
\item Bestimmen Sie die entsprechenden Gausspunktkoordinaten $\xi_i$
\enae

Die Zugehörigen Wichtungsfaktoren $w_i$ lassen sich über folgende Beziehung bestimmen:

\eb
w_i=\int_{-1}^1 l_i(\xi)\ \mrm{d}\xi
\ee

Dabei ist $l_i$ das i-te Lagrangepolynom (Polynomgrad: $n-1$) mit den Gausspunkten $\xi_i$ als Stützstellen.

\enabres
\item Bestimmen Sie die zu den Gausspunktkoordinaten gehörigen Wichtungsfaktoren $w_i$
\enae

\clearpage


% \subsection*{Zusatzinfo: Herleitung Gausspunktbestimmung}

