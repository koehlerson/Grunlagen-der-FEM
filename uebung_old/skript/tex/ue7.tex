\clearpage
\setcounter{page}{1}

\section{Lineares Dreieckselement}

Ziel dieser Übung ist es die benutzereigene FEAPpv Elementsubroutine \verb|elmt03| für lineare Dreieckselemente zu erstellen und in das Programm einzubinden.
Dazu werden in den Teilaufgaben stückweise die Unter-Subroutinen erstellt, mithilfe derer schließlich die Elementsteifigkeitsmatrix des linearen Verschiebungs-Dreieckselementes erzeugt wird.



\subsection*{Kompilieren und Ausführen von Fortran Programmen}

Zum Kompilieren und Ausführen der Beispielhaften Fortran Datei $\tt hello\_world.f $ werden auf dem CIP-Pool Rechner folgende Schritte durchgeführt

\begin{enumerate}[label=\arabic*.)]
 \item den bereitgestellten Ordner $\tt ue07\_f$ im persönlichen Laufwerk Z ablegen %(sollte $\tt /cygdrive/z/$ einen Ordner (z.B. $\tt uebung07$) erstellen
 \item unter alle Programme/Cygwin-X auf \enquote{XWin Server} klicken und im Anschluss durch Rechtsklick auf das Cygwin"=Symbol in der Taskleiste unter Systemwerkzeuge ein Cygwin"=Terminal starten,
 \item mit dem Befehl \verb|cd /cygdrive/z/ue07_f| in den Ordner \verb|ue7_f| navigieren
 \item in der Konsole \verb|gfortran -o hello_world hello_world.f| eingeben. 
 Mit diesem Aufruf wird dem Compiler \verb|gfortran| der Befehl gegeben die ausführbare Datei \verb|hello_world| aus dem Fortran-Quellcode der Datei \verb|hello_world.f| zu kompilieren.
 \item in der Konsole \verb|./hello_world| eingeben um das Progamm auszuführen.
 \item[]
 (ggf. muss bei Windows-Systemen bei den Kommandozeileneingaben die Endung \verb|.exe| bei \verb|gfortran| und \verb|hello_world| hinzugefügt werden.)
\end{enumerate}

Schritte 4 und 5 lassen sich auf jeden Fortran Quellcode \verb|<filename>.f| anwenden.\medskip


\clearpage
\subsection*{Hinweise zum Programmieren mit Fortran}

Zur Programmierung k\"onnen nur die Spalten $6-72$ verwendet werden.\\

% \begin{table}
\begin{tabular}{p{60mm}p{70mm}}
%
\textbf{Befehl}                               & \textbf{Wirkung} \\
%
\midrule
\verb|programm <program name>|\newline
...\newline
...\newline
\verb|end program|                            & erste und letzte Zeile eines Programms\\
%
\midrule
\verb|integer i|\newline
\verb|real*8 a|\newline
\verb|real*8 b(10)|                           & Initialisierung von der Variablen i als Integer, a als reelle Zahl und b als Array mit 10 reellen Einträgen \\
%
\midrule
\verb|do i=1,100|\newline
...\newline
...\newline
\verb|end do|                                 & Schleife, in der die Variable i von 1 bis 100 iteriert wird.\\
%
\midrule
\verb|if(r.lt.0)then|\newline
... \verb|Block A| ...\newline
\verb|else|\newline
... \verb|Block B|...\newline
\verb|end if|                                & \enquote{wenn, dann}-Abfrage; hier: wenn r kleiner als 0 ist, wird Block A ausgeführt, ansonsten Block B \\
 \midrule
% \verb|open(1,file='eps_sig.dat')|\newline
% ...\newline
% \verb|write(1,*)'Hello World'|\newline
% ...\newline
% \verb|close(1)| & Öffnen, beschreiben und schließen einer Datei namens \enquote{eps\underline{ }sig.dat} auf unit 1\\
% \hline
\end{tabular}
% \end{table}



\clearpage

\subsection{Berechnen der Elastizitätsmatrix}

Zur Berechnung der Spannungskomponenten in der x-y-Ebene lautet die Elastizitätsmatrix $\mIC^{(V)}$ in Voigt-Notation:

\eb
\mIC=
\begin{bmatrix}
  \lambda + 2\,\mu & \lambda            & 0\\
  \lambda          & \lambda + 2\,\mu  & 0\\
   0              &   0                & \mu
\end{bmatrix}
\ee

Vervollständigen Sie in der Datei \verb|elmt03.f| die Subroutine \verb|dmat03|, sodass mithilfe der Eingabeparameter {\tt yo} (E-Modul) und {\tt nu} (Querkontraktionszahl) die einzelnen Komponenten der Elastizitätsmatrix {\tt aa} berechnet werden.\medskip

\textit{Hinweis:} Um die in dieser und in den folgenden Aufgaben erzeugten Subroutinen zu testen kann die Datei $\tt t1\_test.f$ verwendet werden.
Diese beinhaltet einen Parametersatz (\verb|t1_param|) für ein beispielhaftes Dreieckselement (vgl. Abbildung \ref{fig:t1exam}) und gibt die mit den Subroutinen berechneten Ergebnisse in der Konsole aus (\verb|t1_output_formats|)

{\center
\begin{minipage}{0.46\textwidth}
\center\input{fig/ue7_t1_example.pdf_tex}
\end{minipage}
\begin{minipage}{0.46\textwidth}
\begin{tabular}{ll}
%   \multicolumn{2}{c}{Parameter}\\
\toprule
 E      & $10000$ MPa \\
 $\nu$  & $0.3$ \\
 $x_1,y_1$ &  $0.0, 0.0$ mm \\
 $x_2,y_2$ &  $1.0, 0.3$ mm \\
 $x_3,y_3$ &  $0.5, 0.8$ mm \\
\midrule
\end{tabular}
\end{minipage}
\captionof{figure}{Knotenkoordinaten und Elastizitätsparameter eines Beispielelementes}
\label{fig:t1exam}}





\subsection{Berechnen der B-Matrix}

Die sogennante B-Matrix $\mB^{\mrm{e}}$ hat für das lineare Dreieckselement die Form

\eb
\mB^{\mrm{e}}=
\begin{bmatrix}
 N_{1,x}  & 0        & N_{2,x}  & 0        & N_{3,x}  & 0  \\
 0        & N_{1,y}  & 0        & N_{2,y}  & 0        & N_{3,y}  \\
 N_{1,y}  & N_{1,x}  & N_{2,y}  & N_{2,x}  & N_{3,y}  & N_{3,x}
\end{bmatrix},
\ee

wobei die Ableitungen der Ansatzfunktionen $N_{I}$ für $I\in\{1,2,3\}$ nach den physikalischen Koordinaten mit folgender Formel aufgestellt werden:

\eb
\begin{bmatrix}
 N_{1,x} \\ N_{2,x} \\ N_{3,x}
\end{bmatrix}= \frac{1}{2 A^{\mrm{e}}}
\begin{bmatrix}
 y_2-y_3 \\ y_3-y_1 \\ y_1-y_2
\end{bmatrix}, \qquad
\begin{bmatrix}
 N_{1,y} \\ N_{2,y} \\ N_{3,y}
\end{bmatrix}= \frac{1}{2 A^{\mrm{e}}}
\begin{bmatrix}
 x_3-x_2 \\ x_1-x_3 \\ x_2-x_1
\end{bmatrix}
\ee

Dabei stellt $A^{\mrm{e}}$ die Elementfläche dar, welche sich mithilfe der Determinante der Transformationsmatrix $\mA^{\mrm{e}}$ berechnen lässt:

\eb
2 A^{\mrm{e}} = \det{\mA^{\mrm{e}}} = (x_2-x_1)(y_3-y_1)+(x_3-x_1)(y_1-y_2) \quad\text{mit}\quad
\mA^{\mrm{e}} = \begin{bmatrix}
       1   & 1   & 1   \\
       x_1 & x_2 & x_3 \\
       y_1 & y_2 & y_3
      \end{bmatrix}
\ee

Vervollständigen Sie in der Datei \verb|elmt03.f| die Subroutine \verb|bmat03|, sodass mithilfe der Eingabeparameter \verb|det| (Elementfläche $2 A^{\mrm{e}}$) und \verb|xl| (Knotenkoordinatenmatrix $[2 \times 3]$) die einzelnen Komponenten der B-Matrix {\tt bmat} berechnet werden.


\subsection{Elementsteifigkeitsmatrix aufstellen}

Für lineare Dreieckselemente lässt sich die Elementsteifigkeitsmatrix $\mk^{\mrm{e}}$ wie folgt berechnen:

\eb
\begin{split}
\mk^e&=
\int_{\B^{\mrm{e}}} (\mB^{\mrm{e}})^T \mIC \mB^{\mrm{e}}\ \mrm{d}V =
A^{\mrm{e}}(\mB^{\mrm{e}})^T \mIC \mB^{\mrm{e}}\\ 
% \quad\text{da } \mB^{\mrm{e}}=const.\\
[6\times 6] &= [6 \times 3][3\times 3][3\times 6]
\end{split}
\ee

Hierzu sei eine Scheibendicke von $h$=1 mm angenommen.
Vervollständigen Sie in der Datei \verb|elmt03.f| die Subroutine \verb|kemat03|, sodass mithilfe der Eingabeparameter \verb|bmat| (B-Matrix), \verb|aa| (Elastizitätsmatrix), \verb|det| ($2 A^{\mrm{e}}$) und \verb|nst| (Dimension von $\mk^{\mrm{e}}$) die einzelnen Komponenten der Elementsteifigkeitsmatrix {\tt s} berechnet werden.\medskip

\textit{Hinweis:} Anders als bei \matl\ (vgl. Übung 5) werden Matrix-Multiplikationen in Fortran nicht automatisch durchgeführt. 
Die einzelnen Komponenten des resultierenden Matrixproduktes müssen per Schleifendurchlauf berechnet werden.
Dazu bietet sich an die jeweiligen Matrixmultiplikationen schrittweise zu berechnen und zunächst das Zwischenprodukt

\ebn
\verb|btc| \defeq (\mB^{\mrm{e}})^T \mIC
\een

auszurechnen, mithilfe dessen daraufhin 

\ebn
\verb|s| \defeq A^{\mrm{e}} (\verb|btc|) \mB^{\mrm{e}}
\een

berechnet werden kann.

\clearpage
\subsection{Einbinden in FEAPpv und Beispielrechnung\label{subsec:t1feap}}

\enab
\item Vervollständigen Sie die Elementsubroutine \verb|elmt03| indem sie die aus den voherigen Teilaufgaben aufgestellten Unter-Subroutinen nacheinander aufrufen um die Elementsteifigkeitsmatrix aufzustellen.
\item Ersetzen Sie die nun vervollständigte Datei \verb|elmt03.f| in dem Dateipfad \verb|$FEAPPVHOME4_1/user| und updaten Sie das FEAPpv-Hauptprogram mit dem Befehl \verb|make|.

(Eventuell ist es notwendig die in dem Verzeichnis befindliche Datei \verb|elmt03.f| dem \verb|$FEAPPVHOME4_1/user|-Dateipfad einmal zu öffnen und zu speichern.)
\item Führen die Cook's Membrane-Problem Rechnung (Input-Datei: $\tt I\_cm$) für einen der Netzverfeinerungzustände aus Aufgabe 6.1a) mit dem soeben erstellten User-Element durch und vergleichen Sie die Lösungen.
\enae

\textit{Hinweis:} Benutzerdefinierte Elemente werden in der FEAPpv Inputdatei durch den Befehl

\begin{verbatim}
mate <number>
  user, <user elmt number>
  <user mat par1>  <user mat par2> ...
\end{verbatim}

aufgerufen.
In dem hier vorliegenden Fall sollten die Ergebnisse des implementierten User-Elementes

\begin{verbatim}
mate 1
  user, 3
  E nu
\end{verbatim}

mit denen des in der vorherigen Aufgabe verwendeten programmeigenen linearen Dreieckselementes 

\begin{verbatim}
mate 1
  soli
  elas isot E nu  
\end{verbatim}

mit den Initialisierungsgrößen $\tt \{ndm,nen\}=\{2,3\}$ übereinstimmen.
