\clearpage
\setcounter{page}{1}

\section{Quadratisches Dreieckselement}


In dieser Übung wird die FEAPpv Elementsubroutine \verb|elmt04| für isoparametrische lineare und quadratische Dreieckselemente vervollständigt. 
Ziel ist es insbesondere die Einträge der B-matrix ($N_{I,x}, N_{I,y}$) zu berechnen mithilfe  der kinematischen Beziehungen zwischen Parameter- und physikalischen Koordinaten (Jacobimatrix $\mJ$, Jacobi-Determinante $\det \mJ$ und Inverse der Jacobimatrix $\mJ^{-1}$).


{\center\begin{minipage}{0.36\textwidth}
\input{fig/ue8_p2.pdf_tex}
\end{minipage}
\begin{minipage}{0.36\textwidth}
\center\input{fig/ue9_t2_example.pdf_tex}
\end{minipage}
\captionof{figure}{Quadratisches Referenzdreieckselement im Parameterraum und Beispielement im physikalischen Raum}
\label{fig:p2isph}
}




\subsection{Ableitung der Ansatzfunktionen}


Die Ansatzfunktionen des quadratischen Dreieckselementes lauten 

\eb
\begin{bmatrix}
 N_1 \\ N_2 \\ N_3 \\ N_4 \\ N_5 \\ N_6
\end{bmatrix}=
\begin{bmatrix}
 \lambda_1 ( 2 \lambda_1 - 1) \\ \lambda_2 ( 2 \lambda_2 - 1) \\ \lambda_3 ( 2 \lambda_3 - 1) \\
 4 \lambda_1 \lambda_2 \\ 4 \lambda_2 \lambda_3 \\ 4 \lambda_3 \lambda_1
\end{bmatrix},\quad\text{wobei sich die Flächenkoordinaten mit}\quad
\begin{bmatrix}
 \lambda_1 \\ \lambda_2 \\ \lambda_3
\end{bmatrix}=
\begin{bmatrix}
 \xi \\ \eta \\ 1-\xi-\eta 
\end{bmatrix}
\ee
durch die Parameterkoordinaten $\xi$ und $\eta$ ausdrücken lassen.
Mithilfe der Ableitungen der Ansatzfunktionen nach den Parameterkoordinaten und den physikalischen Knotenkoordinaten lässt sich mithilfe des isoparametrischen Konzeptes die Jacobimatrix

\eb
\mJ = \sum_{I=1}^{\mrm{nen}}
\begin{bmatrix}
 x_I N_{I,\xi}   &  x_I N_{I,\eta}   \\
 y_I N_{I,\xi}   &  y_I N_{I,\eta}
\end{bmatrix}=
\begin{bmatrix}
 x_1 & x_2 & \hdots & x_{\mrm{nen}}  \\
 y_1 & y_2 & \hdots & y_{\mrm{nen}}
\end{bmatrix}
\begin{bmatrix}
 N_{1,\xi} & N_{1,\eta} \\
 N_{2,\xi} & N_{2,\eta} \\
 \vdots    & \vdots     \\
 N_{\mrm{nen},\xi} & N_{\mrm{nen},\eta}
\end{bmatrix}
\label{eq:jacmat}
\ee



aufstellen. 
In der Datei \verb|elmt04.f| in der Subroutine \verb|shape04| wird die Koordinatenmatrix mit \verb|xl| und die Matrix bestehend aus Ableitungen der Ansatzfunktionen nach Parameterkoordinaten mit \verb|sh0| bezeichnet.
\eqref{eq:jacmat} wird demzufolge berechnent durch \verb|xl|.\verb|sh0|$^T$.


\enab
\item Implementieren Sie \eqref{eq:jacmat} zum Aufstellen der Jacobimatrix in der Subroutine \verb|shape04|.
\enae

Zur Berechnung der Ableitung der Ansatzfunktionen nach den physikalischen Koordinaten wird die Inverse der Jacobimatrix benötigt.
Diese lässt sich im 2D-Fall direkt berechnen mit:

\eb
\mJ^{-1}=\frac{1}{\det \mJ} 
\begin{bmatrix}
 \ J_{22}   & - J_{12} \\
 - J_{21}   & \ J_{11}
\end{bmatrix}
\ee

\enabres
\item Ergänzen Sie \verb|shape04| um die Jacobideterminante und die Inverse der Jacobimatrix.
\enae

Die Ableitungen der Ansatzfunktionen lassen sich mit

\eb
\begin{bmatrix}
 N_{I,x} \\ N_{I,y} 
\end{bmatrix}= \mJ^{-T}
\begin{bmatrix}
  N_{I,\xi} \\ N_{I,\eta}
\end{bmatrix}
\ee

durch das folgende Matrixprodukt aufstellen:

\eb
\begin{bmatrix}
 N_{1,x} & \hdots & N_{\mrm{nen},x} \\
 N_{1,y} & \hdots & N_{\mrm{nen},y}
\end{bmatrix}\defeq
\begin{bmatrix}
 \verb|shp(1,*)| \\ \verb|shp(2,*)|
\end{bmatrix}
=\mJ^{-T}.\verb|sh0|
\label{eq:dNidx}
\ee

\enabres
 \item Vervollständigen Sie \verb|shape04| durch die Berechnung der Ableitungen \eqref{eq:dNidx}
\enae


\textit{Hinweis:} Wie in Übung 7 kann die Datei $\tt t2\_test.f$ verwendet werden um die Datei \verb|elmt04.f| zu testen.
Der zur Tabelle \ref{tab:p2para} gehörige Parametersatz für ein entsprechendes Beispielelement befindet sich in der Datei \verb|t2_param|.% und die Ausgabebefehle in der Datei \verb|t2_output_formats|.



{\center
\begin{tabular}{ll|ll}
%   \multicolumn{2}{c}{Parameter}\\
\toprule 
 E      & \multicolumn{3}{l}{$10000$ MPa} \\
 $\nu$  & \multicolumn{3}{l}{$0.3$}  \\
 $x_1,y_1$ &  $0.0, 0.0$ mm & $x_4,y_4$ &  $0.50, 0.15$ mm \\
 $x_2,y_2$ &  $1.0, 0.3$ mm & $x_5,y_5$ &  $0.75, 0.55$ mm \\
 $x_3,y_3$ &  $0.5, 0.8$ mm & $x_6,y_6$ &  $0.25, 0.40$ mm \\
\midrule
\end{tabular}
\captionof{table}{Parameter des quadratischen Dreiecks-Beispielelementes}
\label{tab:p2para}
}


\clearpage
\subsection{Konvergenzstudie}

\enab
\item Ersetzen Sie die vervollständigte Datei \verb|elmt04.f| in dem Dateipfad \verb|$FEAPPVHOME4_1/user| und updaten Sie das FEAPpv-Hauptprogram mit dem Befehl \verb|make|
(vgl Aufgabe 7.4).%\ref{subsec:t1feap})
\item Führen Sie eine Beispielrechnung mit dem linearen und quadratischen User-Element \verb|elmt04.f| durch und vergleichen sie die Lösungen mit denen der entsprechenden programmeigenenen Dreieckselemente.
\item Führen Sie die Netzkonvergenzstudie aus Aufgabe 6.1a) für das quadratische Dreieckselement durch und vergleichen Sie die Ergebnisse mit denen des linearen Dreieckselementes. 
Welcher der beiden Elementtypen hat hier ein vorteilhafteres Konvergenzverhalten?
\enae
