\clearpage
\setcounter{page}{1}

\textbf{Hinweis: }Benatworten Sie Verständisfragen stichpunktartig. 
Achten Sie darauf, dass Ihre Antworten eindeutig und verständlich sind.


\section{Klausur}

\clearpage
\subsection{Grundidee der FEM und Elementtypen}


\enab
\item Wie lautet das Randwertproblem der Elastostatik (lokale Form)?
\item Wie lautet die zugehörige schwache Form (nur finale Form angeben)? 
      Welche Schritte werden durchgeführt um von dem Randwertproblem der Elastostatik zur schwachen Form zu gelangen? 
% Antw.: Partielle Integration, Gausscher integralsatz, Aufteilen des Randes in Dirichlet/Neumann rand
\item Warum wird die schwache Form für die FE-Diskretisierung verwendet?
% 
\item Was ist die Grundidee der FEM? 
      Beschreiben Sie die einzelnen Terme der Beziehung $\Mvarepsilon^h=\mB_e\,\md_e$ 
\item Nennen Sie drei verschiedene Elementtypen (egal ob 1D,2D oder 3D) und geben sie an ob es sich um $C^0$ oder $C^1$-kontinuierliche Elemente handelt.
      Begründen Sie ihre Antwort.
% Antw.: Diskretisierung, Elementsteifigkeitsmatrizen und Lastvektoren aufstellen, Assemblierung, Lösen des globalen Gleichungssystems


\enae


\clearpage

\subsection{Isoparametrisches Vierknotenelement}

\enab
\item Skizzieren Sie das bilineare Vierknotenelement (mit Knotennummern und -koordinaten) im Parameterraum
\item Was besagt das isoparametrische Konzept?
\item Was ist der Parameterraum und warum wird er verwendet?
% Antw: numerische Integration möglich.. Ansatzfunktionen müssen nur einmal aufgestellt werden.
\item Bestimmen Sie die Ansatzfunktionen des bilinearen Vierknotenelementes.
Nutzen Sie dazu die zugehörigen Lagrangepolynome.
\item Was ist bei der FEM mit dem isoparametrischen Konzept die Jacobi-Determinante und wozu wird sie benötigt?
\enae

\clearpage

\subsection{Numerische Integration}

% \enab
% \item Warum ist es üblich bei dem isoparametrischen Vierknotenelement die $2\times 2$-Gaussintegration zu verwenden? 
% Was passiert wenn eine niedrigere Gaussintegrationsordnung verwendet wird? 
% % Antw: höher: kein Gewinn an genauigkeit, steigender Rechenaufwand
% %      niedriger: einpunkt immer noch exakt. möglicher Rangabfall der globalen Steifigkeitsmatrix
% %      allgemein zu niedrig: Abfall der Konvergenzrate <=> Integrationsfehler > Diskretisierungsfehler 
% \enae

Im Folgenden wird das folgende quadratische Polynom betrachtet:

\eb
f(\xi)=\frac{\xi^2}{4}-\frac{\xi}{4}
\ee

Es soll mithilfe der Gauss-Quadratur über das Intervall $[-1,1]$ numerisch integriert werden.

\enab
\item Wieviele Gausspunkte $n$ sind erforderlich damit die numerische Integration exakt ist? 
 Begründen Sie Ihre Antwort mihtilfe der zugehörigen Bestimmungsformel.
\item Bestimmen Sie die entsprechenden Gausspunktkoordinaten $\xi_i$. Nutzen Sie dazu zu $n$ passende Legendre-Polynom.
\item Bestimmen Sie die zugehörigen Wichtungsfaktoren mithilfe der Formel 
$\displaystyle w_i=\int_{-1}^1 l_i(\xi)\ \mrm{d}\xi $
% Integrieren Sie dazu die zu den Gausspunkten gehörigen Lagrangepolynome über das betrachtete Intervall.
\item Zeigen Sie, dass für die berechneten Gausspunkte $\xi_i$ und Wichtungsfaktoren $w_i$ die Näherung
\ebn
\int_{-1}^1 f(\xi) \ \mrm{d}\xi \approx \sum_{i=1}^n w_i\ f(\xi_i)
\een
     exakt erfüllt ist indem Sie beide Seiten ausrechen.
\enae




\clearpage
\subsection{Assemblierung}

{\center
\input{fig/k1_assem.pdf_tex}
\captionof{figure}{}
 \label{fig:k1assem}
}


\enab
\item Benennen Sie die einzelnen Terme in der Gleichung $\mK\mD=\mP$
\item Was sind Dirichlet- und was sind Neumann Randbedingungen? 
      Was passiert mit der Gleichung aus (a) wenn keine Dirichlet-Randbedingungen vorgegeben werden?
%       Was passiert mit dem Gleichungssystem aus der vorherigen Teilaufgabe wenn keine Dirichlet-RB vorgegeben werden?
\item Erstellen Sie eine Skizze in der alle globalen Knotenfreiheitsgrade (ohne Berücksichtigung der Randbedingungen) eingetragen sind.
\item Wie lauten die Einträge des globalen Lösungsvektors nachdem die zu den Lagerungsrandbedingungen gehörigen Einträge gestrichen wurden? 
      Wie lautet der globale Lastvektor?
\item Wie lauten die Einträge der Elementsteifigkeitsmatrix $\mk^1$ und $\mk^2$, welche nach Berücksichtigung der Randbedingungen in die globale Steifigkeitsmatrix eingehen?
      Bestimmen Sie die globale Steifigkeitsmatrix.
%       Bestimmen Sie die Einträge der globalen Steifigkeitsmatrix $\mK$.
\enae



\clearpage
\subsection{Locking, Konvergenz und Diskretisierungsfehler}

\enab
\item Was versteht man unter einer Netzkonvergenzstudie? Warum wird sie durchgeführt? Skizzieren Sie beispielhaft einen ensprechenden Plot.
\item Nennen Sie drei Probleme, die Verschiebungs-Elementen mit linearen Ansatzfunktionen auftreten können?
      Geben sie an unter welchen Bedingungen diese Probleme auftreten können und wie sie sich äußern (ggf. Skizzen).
\item Welche zwei Ansätze für gemischte Elemente kennen Sie und wie lauten jeweils die zugehörigen Lösungsgrößen?
\item Erläutern sie die Begriffe Diskretisierungsfehler, A-Posteori Fehlerschätzer und Effektivitätsindex.
% Aw: Verhältnis von L2-norm des Fehlerschätzers zu L2-norm des Wahren Fehlers
\enae


