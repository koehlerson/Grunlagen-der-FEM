\clearpage
\setcounter{page}{1}


\section{Klausurvorbereitung}



\subsection{Schwache Form und Idee der FEM}

Gegeben sei das elastische Potential

\ebn
\varPi = \frac{1}{2} \int_{\B} \Bvarepsilon \cdot \Bsigma\ \mrm{dv} 
- \int_{\B} \bu \cdot \bar{\bb}\ \mrm{dv}
- \int_{\p{}\B_t} \bu \cdot \bar{\bt}\ \mrm{da}
\een

\enab
\item Wie lautet die zugehörige Variationsgleichung (schwache Form)?
\item Leiten Sie aus der schwachen Form die Euler Lagrange-Gleichungen (starke Form) für das Randwertproblem der Elastostatik her.
\item Wie gelangt man von der schwachen Form zur Elementsteifigkeitsmatrix $\mk_e$ und Elementlastmatrix $\mmp_e$?
\item Wie lassen sich die Integrale in der diskretisierten schwachen Form berechnen?
\item Wie gelangt man zur Lösung des globalen Problems?
\enae


\subsection{Assemblierung}

Gegeben sei das in Abbildung \ref{fig:kvassem} dargestellte Mini-Randwertproblem bestehend aus linearen Dreieckselementen. 


{\center
\input{fig/kv_assem.pdf_tex}
\captionof{figure}{}
\label{fig:kvassem}
}

\enab
\item Erstellen Sie eine Skizze, in der alle globalen Knotenfreiheitsgrade (ohne Berücksichtigung der Randbedingungen) eingetragen sind.
\item Wie lauten die Einträge des globalen Lösungsvektors unter Berücksichtigung der Randbedingungen? 
      Wie lautet der globale Lastvektor?
\item Wie lauten die Einträge der Elementsteifigkeitsmatrix $\mk^1$ und $\mk^2$, welche nach Berücksichtigung der Randbedingungen in die globale Steifigkeitsmatrix eingehen?
      Bestimmen Sie die Einträge der globalen Steifigkeitsmatrix $\mK$.
\enae



% \subsection{Lagrange Polynome}


\clearpage
\subsection*{Fragenkatalog}

\subsubsection*{Kontinuumsmechanik}
\begin{itemize}
 \item Was ist die Voigt-Notation? Welche Annahmen ermöglichen es diese zu verwenden?
 \item Wie lautet der Elastizitätstensor in Voigt-Notation?
 \item Was ist der Unterschied zwischen ebenem Verzerrungs- und ebenem Spannungszustand?
 % Aw: annahmen: symmetrie von \varepsilon and \sigma
\end{itemize}



% \subsubsection*{Assemblierung}
% \begin{itemize}
%   \item Assemblierungsaufgabe wie in Übung
% %   \item Wie werden Dirichlet-Randbedingungen in dem globalen Gleichungssystem berücksichtigt?
%  \item Wie werden Neumann-Randbedingungen in dem globalen Gleichungssystem berücksichtigt?
%  \item Grundprozedur FEM
%  \item Welche Informationen stecken in der Elementsteifigkeitsmatrix?
% \end{itemize}



\subsubsection*{Grundkonzept der FEM}
\begin{itemize}
 \item Was sind die grundlegenden Schritte der Finite-Elemente Analyse?
 \item Leiten sie Elementsteifigkeitsmatrix und -lastmatrix her
  \item Wie werden Lagerungen in dem globalen Gleichungssystem berücksichtigt, wie werden die ensprechenden Randbedingungen noch genannt?
  \item Welche zwei verschiedenen Arten von Randbedingungen kennen Sie? Worin unterscheiden sie sich?
  \item Was passiert wenn Randbedingungen weggelassen werden?
  \item Welche verschiedenen Elementtypen kennen Sie?
\end{itemize}


\subsubsection*{Isoparametrisches Konzept}
\begin{itemize}
 \item Wie lauten die Ansatzfunktionen für das bilineare/quadratische Viereckselement? (Aufstellen der zugehörigen Lagrange-Polynome)
 \item Was besagt das Isoparametrische Konzept? Warum wird es verwendet?
 \item Wozu wird die Jacobideterminante benötigt? Wie wird sie bestimmt?
\end{itemize}



\subsubsection*{Numerische Integration}
\begin{itemize}
\item Warum wird die Gauss-Integration benötigt?
\end{itemize}


Gegeben sei ein Polynom $f(\xi)$ mit dem Polynomgrad $n=1/2/3/...$. 
\begin{itemize}
 \item Wieviele Gausspunkte sind erforderlich damit die Integration exakt ist?
 \item Wie lauten die zugehörigen Gausspunkte und Wichtungsfaktoren? (entsprechende Formeln auswerten)
 \item Was passiert wenn eine höhere Integrationsordung als erforderlich verwendet wird?
 \item Wie lauten die Legendre-Polynome der Ordnung $n=1/2/3/...$?
\end{itemize}

\subsubsection*{Konvergenz und Diskretisierungsfehler}

\begin{itemize}
 \item Wie lässt sich das Konvgenzverhalten verschiedener Elemente vergleichen? Skizzieren Sie einen beispielhaften entsprechende Konvergenzplot.
 \item Welche zwei verschiedenen Arten von A-Posteori Fehlerschätzern haben Sie kennengelernt, worin unterscheiden sich diese?
 \item Womit lässt sich die Qualität eines Fehlerschätzers ermitteln?
 \item Wird der Diskretisierungsfehler bei Erhöhung der Elementanzahl für das gleiche Randwertproblem größer oder kleiner? Begründen Sie Ihre Antwort.
 \item Was sind superkonvergente Punkte?
 \item Handelt es sich bei den Verschiebungen $\bu^h$, wenn sie mit Lagrange-Ansatzfunktionen über die Knoten interpoliert um Funktionen, welche zwischen den Elementen kontinuierlich sind? Wie verhält es sich mit den zugehörigen Verzerrungen $\Bvarepsilon^h$? 
 \item An welchen Punkten werden Verzerrungen und Spannungen im Postprocessing üblicherweise ausgewertet und warum?
 \end{itemize}


\subsubsection*{Locking und gemischte Elemente}

\begin{itemize}
 \item Wann spricht man von inkompressiblem Material? Welche Probleme können auftreten?
 \item Was sind die Vorteile gemischter Elemente gegenüber standard Verschiebungselementen?
 \item Wie lautet das Hellinger-Reissner Variationsprinzip?
 \item Wie lautet das Hu-Washizu Variationsprinzip für das Q1P0 element?
\end{itemize}


