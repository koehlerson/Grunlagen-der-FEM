\clearpage
\setcounter{page}{1}

\section{Assemblierung}


\subsection{Ebenes Fachwerk\label{subsec:assfw1}}

Gegeben sei das folgende Fachwerk:\medskip

\begin{minipage}[b]{0.4\textwidth}
  \input{fig/ue4_fachwerk1.pdf_tex}
  \captionof{figure}{Randwertproblem}
  \label{fig:fachw1}
\end{minipage}
 \hfill
\begin{minipage}[b]{0.56\textwidth}
  \input{fig/ue4_fachwerk1_diskret.pdf_tex}
  \captionof{figure}{Element- und Knotennummerierung}
  \label{fig:fw1dis}
 \end{minipage}
 



\enab
\item Welche grundlegenden Schritte sind erforderlich um das Problem mithilfe der Finite-Elemente Methode zu lösen?
\enae

Im Folgenden sei angenommen, dass die jeweiligen Elementsteifigkeitsmatrizen $\mk^e$ bekannt sind.

\enabres
\item Assemblieren Sie aus den Elementsteifigkeitsmatrizen die globale Steifigkeitsmatrix $\mK$ unter Berücksichtigung der Dirichlet-Randbedingungen. 
 Stellen Sie den globalen Lastvektor $\mP$ und das globale Gleichungssystem auf.
%(Was würde mit der resultierenden globalen Steifigkeitsmatrix passieren wenn diese nicht berücksichtigt werden würden?)
\enae

Nutzen Sie die in Abbildung \ref{fig:fw1dis} dargestellte Element- und Knotennummerierung. 





\subsection{Ebenes Fachwerk}

Gegeben sei das dargestellte ebene Fachwerk, für das die jeweiligen Elementsteifigkeitsmatrizen $\mk^e$ als bekannt angenommen werden.\medskip


\begin{minipage}[b]{0.44\textwidth}
  \input{fig/ue4_fachwerk2.pdf_tex}
  \captionof{figure}{Randwertproblem}
  \label{fig:fachw2}
\end{minipage}
\hfill
\begin{minipage}[b]{0.5\textwidth}
   \input{fig/ue4_fachwerk2_diskret.pdf_tex}
  \captionof{figure}{Diskretisierung}
  \label{fig:fw2dis}
 \end{minipage}\medskip

Stellen Sie den globalen Lastvektor und die globale Steifigkeitsmatrix unter Berücksichtigung der Dirichlet-Randbedingungen auf.
Nutzen Sie die vorgegebene Diskretisierung.
